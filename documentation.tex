% $Id: documentation.tex,v 1.3 2005-08-31 17:27:26 marcelo Exp $
\documentclass[a4]{article}

%% PREAMBLE

\usepackage{verbatim}
\usepackage{listings}

%% PAGE FORMATTING
\setlength{\hoffset}{-1.0in}
\setlength{\textwidth}{17cm}
\setlength{\headsep}{0cm}
\setlength{\textheight}{25cm}
\setlength{\voffset}{-0.7in}
%\setlength{\headwidth}{\textwidth}
\setlength{\headheight}{13.1pt}

%% CONSOLE STYLE
\lstdefinelanguage{console}{
}

\lstnewenvironment{console}[0]{\lstset{
basicstyle=\small,
tabsize=2,
frame=single,
language=console,
columns=flexible,
extendedchars=true,
numbers=none,
breaklines=true,
breakatwhitespace=true
}}{}


%% MISC ENVIRONMENTS AND FUNCTIONS
\newenvironment{functionlist}{
\begin{description}
}{
\end{description}
}
\newcommand{\function}[1]{\item[#1:] }

\newenvironment{parameterlist}{
\begin{itemize}
    }{
\end{itemize}
}
\newcommand{\parameter}[1]{\item \emph{#1}:}

\newcommand{\projectparameter}{\parameter{$--$project} UNIX name of the project. If none is specified, the project specified when you logged in will
be used. If none was specified at that time, an error will be displayed.}


\begin{document}
\title{GForge Command-line interface (CLI) documentation}
\author{}
\date{}
\maketitle

\section{Installation}

In order to install GForge's commandline interface (CLI), you must have PHP
installed in your local machine. The latest PHP4 version is highly recommended.
Also, PHP must be compiled \emph{without} the $--$disable-cli parameter in order to create the
PHP binaries.

You must check in which directory you have your PHP binary installed. The directory
is the prefix that was specified when compiling PHP (if no prefix was defined, it
defaults to /usr, so the PHP binary is located in /usr/bin/php). You can check in which
directory it is installed by issuing the following command:

\begin{console}
# which php
/usr/bin/php
\end{console}

In this case, the PHP binary is \verb|/usr/bin/php|.

\textbf{NOTE:} If you intend using the CLI for administering a GForge installation via an
HTTPS connection, you must compile PHP with the CURL extension (using the $--$with-curl
option in the \verb|configure| script).

Once you have PHP properly installed, you must download and uncompress the CLI
to a directory in your local machine.

Once you've uncompressed the CLI, you must edit \verb|$PREFIX/gforge.php| (where
\verb|$PREFIX| is the directory in which the CLI was uncompressed) using your favorite
editor and change the following:
\begin{enumerate}
    \item Change the first line which says \\
    \verb|#!/usr/bin/php -q| \\
    to make it point to the path of the PHP binary in your machine (if it is different
    from \verb|/usr/bin/php|)
    \item Find the Configuration section in the script (marked between comments) and
    change the following constants:
    \begin{itemize}
        \item \verb|NUSOAP_DIR|: Points to the installation directory of the NuSOAP PHP
        Library. A copy of the library is located in \verb|$PREFIX/nusoap/lib/|.
        \item \verb|GFORGE_CLI_DIR|: Points to the directory where the CLI auxiliary files
        are to be found. It usually is \verb|$PREFIX/include/|.
        \item \verb|WSDL_URL| (optional): URL of the WSDL of your GForge's SOAP server.
        It is probably located in http://(your\_gforge\_server)/soap/index.php?wsdl.
        You can see if the URL is correct by entering it in your browser, you should
        get an XML file from the server as the result. If this variable isn't set, the CLI
        will try to use the WSDL specified in the \verb|GFORGE_WSDL| environment variable.
    \end{itemize}
    \item Make sure the gforge.php script has execution permissions. If not, set the
    permission by issuing the following command:\\
    \verb|chmod +x $PREFIX/gforge.php|
\end{enumerate}

After everything is properly set up, you must test the installation works correctly. For
this, try logging into the server by issuing a ''login'' command. For this, change your
current working directory to the directory in which you uncompressed the CLI and execute
the following:
\begin{console}
[marcelo@linux gforge_cli]$ ./gforge.php -v login --username=(your user) --password=(your password)
GForgeSOAP::Connecting to the server http://gforge/soap/index.php?wsdl...
GForgeSOAP::Connected!
GForgeSOAP::Executing command login...
Logged in as user (your user), using session string MTAyLSotMTEyMzA4Mjg5My0qLTEyNy4wLjAuMS0qLU51U09BUC8wLjYuOSAoMS4xLjEuMSk=-*-824c9f9010931f7872f4b9826c460ed3
Logged in.
\end{console}

The message you see may vary depending on the configuration of your server.

If you get an error like \textit{''Fatal error: wsdl error: HTTP ERROR: Couldn't open socket connection to server http://(your server) prior to connect().''},
it means that your computer can't connect to the GForge server. In this case, you
must make sure the address is written correctly and that you can access the server from your computer.

\section{Usage}
The basic form for invoking the CLI is the following: \\
\verb|# ./gforge.php [program options] [module name] [function name] [function parameters]|

\begin{itemize}
    \item \textbf{Program options:}
    \begin{parameterlist}
\parameter{-v} Verbose. Give a detailed output of what's going on.
\parameter{-h} Help. Display help on how to use the program.
\end{parameterlist}
\item \textbf{Module name:} A ''module'' is a grouping of functions that act over a specific part of the GForge
system. Currently, the following modules are available:
\begin{itemize}
    \item (none): If the module is empty, system-wide functions are accesible (i.e., login to the server)
    \item \verb|frs|: File Release System module
    \item \verb|project|: Module that deals with project administration
    \item \verb|task|: Task management module
    \item \verb|tracker|: Tracker management module
\end{itemize}
\item \textbf{Function name:} The name of the function to invoke depends on the module being called. For details, see
section \ref{parametros}.
\item \textbf{Function parameters:} The parameters also depend on the function being called. See section \ref{parametros}.
\end{itemize}

\section{Module functions and parameters}\label{parametros}

\subsection{System wide functions}
These functions do not act over a specific module. They are the following:
\begin{functionlist}
\function{login}
Log in to the server
\begin{parameterlist}
\parameter{$--$username*\footnote{An asterisk (*) means that the parameter is required}} Specifies the name of the
user you'll be working with. The username is the one you usually use in the GForge login page.
\parameter{$--$password} Specifies the password to login with. If none is provided, you will be asked for one
interactively.
\parameter{$--$project} UNIX name of the project you plan working on. If you don't specify this, any future function
call will require you to specify the project in which the function will be applied.
\end{parameterlist}

Example:
\begin{console}
[marcelo@linux gforge_cli]$ ./gforge.php login --username=joe --password=sixpack --project=myproject
Logged in.

(OR)

[marcelo@linux gforge_cli]$ ./gforge.php login --username=walrus --project=myproject
Password: *******
Logged in.
\end{console}

\function{logout} End a session. This removes all the information of the session from your computer.

Example:
\begin{console}
[marcelo@linux gforge_cli]$ ./gforge.php logout
Session terminated.
\end{console}
\end{functionlist}

\subsection{FRS (File Release System)}
\begin{functionlist}
\function{packages}
Retrieve a list of the available FRS packages for a project.
\begin{parameterlist}
\projectparameter
\end{parameterlist}

Example:
\begin{console}
[marcelo@linux gforge_cli]$ ./gforge.php frs packages
Array
(
    [0] => Array
        (
            [package_id] => 4
            [name] => myproject
            [status_id] => 1
            [is_public] => 1
        )

    [1] => Array
        (
            [package_id] => 8
            [name] => package from the CLI
            [status_id] => 1
            [is_public] => 1
        )

    [2] => Array
        (
            [package_id] => 9
            [name] => package from the CLI 2
            [status_id] => 1
            [is_public] => 0
        )
)
\end{console}

\function{addpackage}
Adds a new package to the FRS.
\begin{parameterlist}
\parameter{$--$name*} Name of the new package.
\parameter{$--$public} Whether the package should be public or not (''y'' or ''n'', default=''y'')
\projectparameter
\end{parameterlist}
Example:
\begin{console}
[marcelo@linux gforge_cli]$ ./gforge.php frs addpackage --name="New package added from the CLI" --public=n
11
\end{console}
In this case, ''11'' (the value outputted) is the ID of the new package.

\function{releases}
Show the releases of a particular package
\begin{parameterlist}
\parameter{$--$package*} ID of the package
\projectparameter
\end{parameterlist}
Example:
\begin{console}
[marcelo@linux gforge_cli]$ ./gforge.php frs releases --package=4
Array
(
    [0] => Array
        (
            [release_id] => 5
            [name] => release #1
            [notes] => this are the release notes
            [changes] => this is the change log
            [status_id] => 1
            [release_date] => 1119884760
        )

    [1] => Array
        (
            [release_id] => 6
            [name] => Release #2
            [notes] => release notes for package #2
            [changes] => change log for package #2
            [status_id] => 1
            [release_date] => 1119966660
        )
)
\end{console}

\function{addrelease}
Add a release to a package
\begin{parameterlist}
\parameter{$--$name*} Name of the new release
\parameter{$--$package*} ID of the package the release will be created in
\projectparameter
\end{parameterlist}
Example:
\begin{console}
[marcelo@linux gforge_cli]$ ./gforge.php frs addrelease --package=11 --name="Release 0.9.1"
12
\end{console}

\function{files}
List the files that are available in a particular release
\begin{parameterlist}
\parameter{$--$package*} ID of the package the release belongs to
\parameter{$--$release*} ID of the release
\projectparameter
\end{parameterlist}
Example:
\begin{console}
[marcelo@linux gforge_cli]$ ./gforge.php frs files --package=4 --release=7
Array
(
    [0] => Array
        (
            [file_id] => 9
            [name] => somefile.tex
            [size] =>
            [type] => Other
            [processor] => Other
            [downloads] => 1
            [release] => 1120149267
            [date] =>
        )
)
\end{console}

\function{addfile}
Add a file to a release
\begin{parameterlist}
\parameter{$--$package*} ID of the package the release belongs to
\parameter{$--$release*} ID of the release
\parameter{$--$file*} File to add
\parameter{$--$type} Type ID of the file. The available types are listed on the \verb|frs_filetype| table 
in the GForge database.
\parameter{$--$processor} Processor ID for this file. The available processors are listed on the \verb|frs_processor| table
in the GForge database.
\parameter{$--$date} Release date. The date must be in format YYYY-MM-DD.
\projectparameter
\end{parameterlist}
Example:
\begin{console}
[marcelo@linux gforge_cli]$ ./gforge.php frs addfile --package=11 --release=12 --file=/home/marcelo/somediff.diff --type=5900 --processor=8000 --date=2005-08-03
16
\end{console}

\function{getfile}
Retrieve a file from the project
\begin{parameterlist}
\parameter{$--$package*} ID of the package the release belongs to
\parameter{$--$release*} ID of the release
\parameter{$--$id*} ID of the file to download
\parameter{$--$output} Name of the file to write the downloaded file to. If none is specified, the file is outputted
to the screen
\projectparameter
\end{parameterlist}
Example:
\begin{console}
[marcelo@linux gforge_cli]$ ./gforge.php frs getfile --package=11 --release=12 --id=16 --output=/home/marcelo/tmp.diff
File retrieved successfully.
\end{console}

\end{functionlist}

\subsection{Project management}
\begin{functionlist}
\function{list}
Display the list of public projects.\\
Example:
\begin{console}
[marcelo@linux gforge_cli]$ ./gforge.php project list
Array
(
    [0] => svntest
    [1] => template
    [2] => phpexplorer
)
\end{console}

\function{mylist}
Display the list of projects assigned to the user you're logged in with.\\
Example:
\begin{console}
[marcelo@linux gforge_cli]$ ./gforge.php project mylist
Array
(
    [0] => Array
        (
            [group_id] => 7
            [group_name] => svntest
            [homepage] => svntest.gforge
            [is_public] => 1
            [status] => A
            [unix_group_name] => svntest
            [short_description] => We will use this project to test the SVN plugin and stuff.
            [scm_box] => gforge
            [register_time] => 1117735073
        )

    [1] => Array
        (
            [group_id] => 5
            [group_name] => PHP Explorer
            [homepage] => phpexplorer.gforge
            [is_public] => 1
            [status] => A
            [unix_group_name] => phpexplorer
            [short_description] => This tool allows you to browse the server's files
            [scm_box] => phpexplorer.gforge
            [register_time] => 1113835609
        )
)

\end{console}

\end{functionlist}

\subsection{Task management}
\begin{functionlist}
\function{groups}
List the task groups in a project
\begin{parameterlist}
\projectparameter
\end{parameterlist}
Example:
\begin{console}
[marcelo@linux gforge_cli]$ ./gforge.php task groups
Array
(
    [0] => Array
        (
            [group_project_id] => 8
            [group_id] => 5
            [name] => To Do
            [description] => Things We Have To Do
            [is_public] => 1
            [send_all_posts_to] =>
        )

    [1] => Array
        (
            [group_project_id] => 9
            [group_id] => 5
            [name] => Next Release
            [description] => Items For Our Next Release
            [is_public] => 0
            [send_all_posts_to] =>
        )
)
\end{console}

\function{list}
List the tasks in a particular group.
\begin{parameterlist}
\parameter{$--$group*} ID of the task group
\parameter{$--$assigned\_to} User ID. Specify this if you want to see all the tasks assigned to a certain user.
\parameter{$--$category} Category ID.
\parameter{$--$status} Status (''open'' or ''closed''). If ommited, it returns all the tasks.
\projectparameter
\end{parameterlist}
Example:
\begin{console}
[marcelo@linux gforge_cli]$ ./gforge.php task list --group=9 --assigned_to=102 --category=100 --status=open
Array
(
    [0] => Array
        (
            [project_task_id] => 14
            [group_project_id] => 9
            [summary] => New item
            [details] => This is the description
            [percent_complete] => 30
            [priority] => 3
            [hours] => 10
            [start_date] => 1119545944
            [end_date] => 1120150744
            [status_id] => 1
            [category_id] => 100
            [dependent_on] => Array
                (
                    [0] => 6
                )

            [assigned_to] => Array
                (
                    [0] => 102
                )

        )
)
\end{console}

\function{categories}
List the available categories for the tasks belonging to a specific group.
\begin{parameterlist}
\parameter{$--$group*} ID of the group.
\projectparameter
\end{parameterlist}
Example:
\begin{console}
[marcelo@linux gforge_cli]$ ./gforge.php task categories --group=9
Array
(
    [0] => Array
        (
            [category_id] => 101
            [category_name] => Category 1
        )

    [1] => Array
        (
            [category_id] => 102
            [category_name] => Category 2
        )

    [2] => Array
        (
            [category_id] => 103
            [category_name] => Category 3
        )
)
\end{console}

\function{add}
Add a new task.
\begin{parameterlist}
\parameter{$--$group*} ID of the task group.
\parameter{$--$summary*} Summary for the task.
\parameter{$--$details*} Details for the task.
\parameter{$--$priority} Priority (1: lowest priority, 5: highest priority). If none is specified, it defaults to 3.
\parameter{$--$hours} Amount of hours assigned to this task.
\parameter{$--$start\_date} Start date for the task. If none is specified, it defaults to today.
\parameter{$--$end\_date} End date for the task. If none is specified, it defaults to a week after the start date.
\parameter{$--$category} Category ID of the task.
\parameter{$--$percent} Percent completed for this task (default: 0).
\parameter{$--$assigned\_to} List of comma-separated user IDs this task is assigned to.
\parameter{$--$dependent\_on} List of comma-separated task IDs this task depends on.
\projectparameter
\end{parameterlist}
Example:
\begin{console}
[marcelo@linux gforge_cli]$ ./gforge.php task add --group=8 --summary="Optimize the SQL commands" --details="There are some slow queries that need to be optimized" --priority=4 --hours=15 --start_date=2005-08-27 --end_date=2005-09-02 --category=105 --percent=10 --assigned_to=102,103 --dependent_on=17
Confirm you want to add a new tracker with the following information:
Project: PHP Explorer
Group: To Do
Summary: Optimize the SQL commands
Priority: 4
Estimated hours: 15
Start date: 2005-08-27
End date: 2005-09-02
Category: Category 2
% complete: 10%
Assigned to: Joe Sixpack (joesix), John Smith (jsmith)
Dependent on: Update database driver
Details:
There are some slow queries that need to be optimized
Is this information correct? (y/n): y
18
\end{console}

\function{update}
Update the information of a task.
\begin{parameterlist}
\parameter{$--$group*} The group the task belongs to.
\parameter{$--$id*} ID of the task to be modified.
\parameter{$--$summary} Summary.
\parameter{$--$details} Details.
\parameter{$--$priority} Priority (1: lowest priority, 5: highest priority)
\parameter{$--$hours} Amount of hours.
\parameter{$--$start\_date} Start date.
\parameter{$--$end\_date} End date.
\parameter{$--$category} Category ID.
\parameter{$--$percent} Percent complete.
\parameter{$--$status} Status (''open'' or ''closed'')
\parameter{$--$assigned\_to} Comma-separated list of user IDs the task is assigned to. Note that the old list gets overwritten if a 
new list is specified.
\parameter{$--$dependent\_on} Comma-separated list of task IDs the task depends on. Note that the old list gets overwritten if a 
new list is specified.
\projectparameter
\end{parameterlist}
Example:
\begin{console}
[marcelo@linux gforge_cli]$ ./gforge.php task update --group=8 --id=18 --percent=100 --status=closed
Confirm you want to update the task with the following information:
Project: PHP Explorer
Group: To Do
Task summary: Optimize the SQL commands
> Percent complete: 100%
> Status: Closed
Is this information correct? (y/n): y
18
\end{console}


\end{functionlist}

\subsection{Tracker management}

\begin{functionlist}
\function{typelist}
Display a list of the trackers for the project.
\begin{parameterlist}
\projectparameter
\end{parameterlist}
Example:
\begin{console}
[marcelo@linux gforge_cli]$ ./gforge.php tracker typelist --group=gforge
Array
(
    [0] => Array
        (
            [group_artifact_id] => 113
            [group_id] => 5
            [name] => Bugs
            [description] => Bug Tracking System
            [is_public] => 1
            [allow_anon] => 0
            [due_period] => 2592000
            [datatype] => 1
            [status_timeout] => 1209600
            [extra_fields] => Array
(...list of trackers and fields...)
\end{console}

\function{list}
Display a list of artifacts that belong to a specified tracker.
\begin{parameterlist}
\parameter{$--$type*} ID of the tracker.
\parameter{$--$assigned\_to} Show only the artifacts assigned to this user ID.
\parameter{$--$status} Show only the artifacts with this status ID.
\projectparameter
\end{parameterlist}
Example:
\begin{console}
[marcelo@linux gforge_cli]$ ./gforge.php tracker list --type=113 --status=0 --assigned_to=102
Array
(
    [0] => Array
        (
            [artifact_id] => 5
            [group_artifact_id] => 113
            [status_id] => 2
            [priority] => 3
            [submitted_by] => 102
            [assigned_to] => 102
            [open_date] => 1115311786
            [close_date] => 1125064392
            [summary] => New bug
            [details] => small bug
            [extra_fields] => Array
                (
                    [0] => Array
                        (
                            [extra_field_id] => 128
                            [field_data] => 40
                        )

                    [1] => Array
                        (
                            [extra_field_id] => 129
                            [field_data] => 41
                        )

                    [2] => Array
                        (
                            [extra_field_id] => 130
                            [field_data] => 42
                        )

                )

        )

)
\end{console}

\function{add}
Add a new artifact to a tracker
\begin{parameterlist}
\parameter{$--$type*} Tracker ID the new artifact will belong to.
\parameter{$--$priority} Priority of the artifact (1: lowest, 5: highest). If not specified, defaults
to 3.
\parameter{$--$assigned\_to} Comma-separated list of user IDs this artifact will be assigned to.
\parameter{$--$summary} Summary for the artifact.
\parameter{$--$details} Details for the artifact.
\parameter{$--$[extra\_field\_name]} Extra field value. \verb|extra_field_name| must be replaced by the
	\emph{alias} of the extra field. The value of the extra field depends on its type.
\projectparameter
\end{parameterlist}
Example:
\begin{console}
[marcelo@linux gforge_cli]$ ./gforge.php tracker add --type=116 --priority=4 --assigned_to=102 --summary="Back button is not working" --details="When I read my email and I press the back button, a blank page appears" --version=52
Confirm you want to add a new tracker with the following information:
Project: PHP Explorer
Tracker type: Feature Requests
Priority: 4
Assigned to: Marcelo Mottalli (walrus)
Summary: Back button is not working
Details:
When I read my email and I press the back button, a blank page appears
Is this information correct? (y/n): y
26
\end{console}
\emph{\textbf{Note}}: In this example, ''version'' corresponds to an extra field and 52
	is a valid value ID for this extra field. The value returned (26) is the ID of the new artifact.

\function{update}
Update an artifact information
\begin{parameterlist}
\parameter{$--$type*} Tracker ID the new artifact will belong to.
\parameter{$--$id*} ID of the artifact to update.
\parameter{$--$priority} Priority of the artifact (1: lowest, 5: highest). If not specified, defaults
to 3.
\parameter{$--$assigned\_to} Comma-separated list of user IDs this artifact will be assigned to.
\parameter{$--$status} New status of the artifact (1: open, 2: closed)
\parameter{$--$summary} Summary for the artifact.
\parameter{$--$details} Details for the artifact.
\parameter{$--$[extra\_field\_name]} Extra field value. \verb|extra_field_name| must be replaced by the
	\emph{alias} of the extra field. The value of the extra field depends on its type.
\projectparameter
\end{parameterlist}
Example:
\begin{console}
[marcelo@linux gforge_cli]$ ./gforge.php tracker update --type=116 --id=26 --priority=5 --extra_field_notice_textfield="Increased priority"
Confirm you want to update the tracker with the following information:
Project: PHP Explorer
Tracker type: Feature Requests
Summary: Back button is not working
> Priority: 5
Is this information correct? (y/n): y
26
\end{console}

\function{messages}
Retrieve all the messages posted in an artifact.
\begin{parameterlist}
\parameter{$--$type*} Tracker ID.
\parameter{$--$id*} Artifact ID.
\projectparameter
\end{parameterlist}
Example:
\begin{console}
[marcelo@linux gforge_cli]$ ./gforge.php tracker messages --type=116 --id=3
Array
(
    [0] => Array
        (
            [id] => 22
            [artifact_id] => 3
            [body] => we will keep them gray
            [adddate] => 1117810766
            [user_id] => 102
        )

    [1] => Array
        (
            [id] => 21
            [artifact_id] => 3
            [body] => maybe we can add a gradient
            [adddate] => 1117809234
            [user_id] => 102
        )

    [2] => Array
        (
            [id] => 20
            [artifact_id] => 3
            [body] => i think most buttons should be green
            [adddate] => 1117723406
            [user_id] => 102
        )

)
\end{console}

\function{addmessage}
Post a new message in an artifact.
\begin{parameterlist}
\parameter{$--$type*} Tracker ID the artifact belongs to.
\parameter{$--$id*} Artifact ID.
\parameter{$--$message*} Message to post.
\projectparameter
\end{parameterlist}
Example:
\begin{console}
[marcelo@linux gforge_cli]$ ./gforge.php tracker addmessage --type=116 --id=3 --message="I agree with you"
47
\end{console}
\emph{\textbf{Note}}: The value returned (47) is the ID of the new message.

\function{files}
Show a list of files attached to the artifact.
\begin{parameterlist}
\parameter{$--$type*} Tracker ID.
\parameter{$--$id*} Artifact ID.
\projectparameter
\end{parameterlist}
Example:
\begin{console}
[marcelo@linux gforge_cli]$ ./gforge.php tracker files --type=116 --id=3
Array
(
    [0] => Array
        (
            [id] => 11
            [artifact_id] => 3
            [name] => screenshot.jpg
            [description] => 11
            [filesize] => 364983
            [filetype] => image/jpeg
            [adddate] => 1124372802
            [submitted_by] => 102
        )

)
\end{console}

\function{getfile}
Download a file from an artifact.
\begin{parameterlist}
\parameter{$--$type*} Tracker ID.
\parameter{$--$id*} Artifact ID.
\parameter{$--$file\_id*} ID of the file to download.
\parameter{$--$output} File name to save the file as. If not specified, the contents of the file are
outputted to the screen.
\projectparameter
\end{parameterlist}
Example:
\begin{console}
[marcelo@linux gforge_cli]$ ./gforge.php tracker getfile --type=116 --id=3 --file_id=11 --output=/home/marcelo/screenshot.jpg
File retrieved successfully.
\end{console}

\function{addfile}
Attach a file to an artifact.
\begin{parameterlist}
\parameter{$--$type*} Tracker ID.
\parameter{$--$id*} Artifact ID.
\parameter{$--$file*} Local file to attach.
\parameter{$--$description} Description of the file.
\projectparameter
\end{parameterlist}
Example:
\begin{console}
[marcelo@linux gforge_cli]$ ./gforge.php tracker addfile --type=116 --id=3 --file=/home/marcelo/image_patch.diff --description="This is the patch to apply"
12
\end{console}
\emph{\textbf{Note}}: The value returned (12) is the ID of the new file.

\function{tecnhicians}
Get a list of technicians assigned to a tracker.
\begin{parameterlist}
\parameter{$--$type*} Tracker ID.
\projectparameter
\end{parameterlist}
Example:
\begin{console}
[marcelo@linux gforge_cli]$ ./gforge.php tracker technicians --type=116
Array
(
    [0] => Array
        (
            [user_id] => 102
            [user_name] => walrus
            [title] => Mr
            [firstname] => Marcelo
            [lastname] => Mottalli
            [address] =>
            [address2] =>
            [phone] =>
            [fax] =>
            [status] => A
            [timezone] => US/Alaska
            [country_code] =>
            [add_date] => 1113835347
            [language_id] => 1
        )

)
\end{console}

\end{functionlist}

\end{document}
